\documentclass[a4paper]{article}
\usepackage[document]{ragged2e}
\usepackage{fontspec}
\usepackage{geometry}
\usepackage{graphicx}
\geometry{left=2cm}
\geometry{right=2cm}
\geometry{top=2cm}
\geometry{bottom=2cm}
\setmainfont{Liberation Serif}
\setmonofont[Scale=0.8]{Hack}
\sloppy

\graphicspath{ {./pics/} }

\begin{document}
  \fontsize{14}{16}\selectfont

  \begin{titlepage}
    \begin{minipage}{0.2\textwidth}
      \includegraphics[scale=0.4]{logo}
    \end{minipage}
    \begin{minipage}{0.7\textwidth}\centering
      \fontsize{10}{12}\selectfont
      \textbf{
        Федеральное государственное бюджетное образовательное учреждение \\
        высшего профессионального образования \\
        «Московский государственный технический университет имени Н.Э. Баумана» \\
        (МГТУ им. Н.Э. Баумана)
      }
    \end{minipage}

    \vspace{5cm}
    \centering
    \textbf{
      Лабораторная работа 1 \\
      по курсу «Технологии машинного обучения» \\
    }

    \vspace{5cm}
    \begin{flushright}
    Выполнил \\
    студент группы ИУ5-64 \\
    XXX
    \end{flushright}
    \vspace*{\fill}
    Москва, 2021
  \end{titlepage}

  \section*{Цель работы}
  Изучение различных методов визуализации данных

  \section*{Задание}
  \begin{itemize}
    \item Выбрать набор данных (датасет).
    \item Создать ноутбук, который содержит следующие разделы:
    \begin{enumerate}
      \item Текстовое описание выбранного Вами набора данных.
      \item Основные характеристики датасета.
      \item Визуальное исследование датасета.
      \item Информация о корреляции признаков.
    \end{enumerate}
    \item Сформировать отчет и разместить его в своем репозитории на github.
  \end{itemize}

  \section*{Ход работы}
  \begin{enumerate}
    \item Текстовое описание набора данных \\
    В качестве набора данных мы будем использовать набор данных о криптовалюте Bitcoin
    (\texttt{https://github.com/Yrzxiong/Bitcoin-Dataset}).

    Датасет состоит из файла \texttt{bitcoin.csv}

    Файл содержит следующие колонки:
    \begin{itemize}
    \item date
    \item market\_price
    \item total\_bitcoins
    \item market\_cap
    \item difficulty
    \item transaction\_fees
    \item n\_unique\_addresses
    \item n\_transactions\_total
    \end{itemize}

    \item Основные характеристики датасета \\
      Описано в секции "2. Основные характеристики датасета" файла 1.ipynb
    \item Визуальное исследование датасета \\
      Описано в секции "3. Визуальное исследование датасета" файла 1.ipynb
    \item Информация о корреляции признаков \\
      Описано в секции "4. Информация о корреляции признаков" файла 1.ipynb
  \end{enumerate}
\end{document}
