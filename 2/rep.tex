\documentclass[a4paper]{article}
\usepackage[document]{ragged2e}
\usepackage{fontspec}
\usepackage{geometry}
\usepackage{graphicx}
\usepackage{url}
\geometry{left=2cm}
\geometry{right=2cm}
\geometry{top=2cm}
\geometry{bottom=2cm}
\setmainfont{Liberation Serif}
\setmonofont[Scale=0.8]{Hack}
\sloppy

\graphicspath{ {./pics/} }

\begin{document}
  \fontsize{14}{16}\selectfont

  \begin{titlepage}
    \begin{minipage}{0.2\textwidth}
      \includegraphics[scale=0.4]{logo}
    \end{minipage}
    \begin{minipage}{0.7\textwidth}\centering
      \fontsize{10}{12}\selectfont
      \textbf{
        Федеральное государственное бюджетное образовательное учреждение \\
        высшего профессионального образования \\
        «Московский государственный технический университет имени Н.Э. Баумана» \\
        (МГТУ им. Н.Э. Баумана)
      }
    \end{minipage}

    \vspace{5cm}
    \centering
    \textbf{
      Лабораторная работа 2 \\
      по курсу «Технологии машинного обучения» \\
    }

    \vspace{5cm}
    \begin{flushright}
    Выполнил \\
    студент группы ИУ5-64 \\
    XXX
    \end{flushright}
    \vspace*{\fill}
    Москва, 2021
  \end{titlepage}

  \section*{Цель работы}
  изучение способов предварительной обработки данных для дальнейшего формирования моделей.

  \section*{Задание}
  \begin{enumerate}
    \item
      Выбрать набор данных (датасет), содержащий категориальные признаки и пропуски в данных.
      Для выполнения следующих пунктов можно использовать несколько различных наборов данных
      (один для обработки пропусков, другой для категориальных признаков и т.д.)
    \item
      Для выбранного датасета (датасетов) на основе материалов лекции решить следующие задачи:

      \begin{itemize}
        \item обработку пропусков в данных;
        \item кодирование категориальных признаков;
        \item масштабирование данных.
      \end{itemize}
  \end{enumerate}

  \section*{Ход работы}
  Описано в файле 2.ipynb
\end{document}
