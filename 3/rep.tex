\documentclass[a4paper]{article}
\usepackage[document]{ragged2e}
\usepackage{fontspec}
\usepackage{geometry}
\usepackage{graphicx}
\usepackage{url}
\geometry{left=2cm}
\geometry{right=2cm}
\geometry{top=2cm}
\geometry{bottom=2cm}
\setmainfont{Liberation Serif}
\setmonofont[Scale=0.8]{Hack}
\sloppy

\graphicspath{ {./pics/} }

\begin{document}
  \fontsize{14}{16}\selectfont

  \begin{titlepage}
    \begin{minipage}{0.2\textwidth}
      \includegraphics[scale=0.4]{logo}
    \end{minipage}
    \begin{minipage}{0.7\textwidth}\centering
      \fontsize{10}{12}\selectfont
      \textbf{
        Федеральное государственное бюджетное образовательное учреждение \\
        высшего профессионального образования \\
        «Московский государственный технический университет имени Н.Э. Баумана» \\
        (МГТУ им. Н.Э. Баумана)
      }
    \end{minipage}

    \vspace{5cm}
    \centering
    \textbf{
      Лабораторная работа 3 \\
      по курсу «Технологии машинного обучения» \\
    }

    \vspace{5cm}
    \begin{flushright}
    Выполнил \\
    студент группы ИУ5-64 \\
    XXX
    \end{flushright}
    \vspace*{\fill}
    Москва, 2021
  \end{titlepage}

  \section*{Цель работы}
  изучение способов подготовки выборки и подбора гиперпараметров на примере метода ближайших соседей.

  \section*{Задание}
  \begin{enumerate}
    \item Выберите набор данных (датасет) для решения задачи классификации или регрессии.
    \item С использованием метода train\_test\_split разделите выборку на обучающую и тестовую.
    \item
      Обучите модель ближайших соседей для произвольно заданного гиперпараметра K.
      Оцените качество модели с помощью подходящих для задачи метрик.
    \item
      Произведите подбор гиперпараметра K с использованием GridSearchCV и/или RandomizedSearchCV
      и кросс-валидации, оцените качество оптимальной модели.
      Желательно использование нескольких стратегий кросс-валидации.
    \item Сравните метрики качества исходной и оптимальной моделей.
  \end{enumerate}

  \section*{Ход работы}
  Описано в файле 3.ipynb
\end{document}
